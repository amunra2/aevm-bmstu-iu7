\chapter{Выполнение лабораторной работы}
\textbf{Система на кристалле  (SoC,  СНК)}  — это функционально законченная электронная вычислительная система, состоящая из одного или нескольких микропроцессорных модулей,а также системных и периферийных контроллеров, выполненная на одном кристалле.  Такая тесная интеграция компонентов системы позволяет достичь высокого быстродействия при построении специализированных ЭВМ.


\section{Схема разрабатываемой СНК}

Кристал на основе ПЛИС Altera представлен на рисунке \ref{img:schem}

\imgScale{0.5}{schem}{Функциональная схема разрабатываемой системы на кристалле}

Система на кристалле состоит из следующих блоков:
\begin{itemize}
	\item Микропроцессорное ядро Nios II/e выполняет функции управления системой.
	\item Внутренняя оперативная память СНК, используемая для хранения программы управления и данных.
	\item Системная шина Avalon обеспечивает связность всех компонентов системы.
	\item Блок синхронизации и сброса обеспечивает обработку входных сигналов сброса и синхронизации и распределение их в системе. Внутренний сигнал сброса синхронизирован и имеет необходимую для системы длительность.
	\item Блок идентификации версии проекта обеспечивает хранение и выдачу уникального идентификатора версии, который используется программой управления при инициализации системы.
	\item Контроллер UART обеспечивает прием и передачу информации по интерфейсу RS232.
\end{itemize}


\section{Проектирование СНК в Quartus 2}

На рисунке \ref{img:module} представлен модуль системы на кристалле в программном обеспечении Quartus 2. Полученная модель соотносится со схемой на рисунке \ref{img:schem} -- добавляются элементы, а также связи между ними через шину \textit{Avalon}.

\imgScale{0.3}{module}{Готовый модуль в системе проектирования}

Quartus 2 выделяет автоматически каждому подключенному компоненту свое собственное адресное пространство (оно единое для данных и кода -- принцип Фон Неймона). Нужно это для того, чтобы адресное пространство было корректно распределено во избежание возникновения ошибок.
На рисунке \ref{img:table} представлена таблица распределния адресов, которая была автоматически получена для дайнной системы.

\imgScale{0.4}{table}{Таблица распределения адресов}

\clearpage

\section{Написание программы}

Для демострации работы предлагалось написать программу, которая, используя специальную функцию (\textit{ORD\_ALTERA\_AVALON\_SYSID\_QSYS\_ID(SYSID\_QSYS\_0\_BASE)}, где \textit{ SYSID\_QSYS\_0\_BASE} — базовый адрес блока SystemID) для получения SystemID номера, выводит его на экран. Сам параметр SystemID был ранее задан при работе с Quartus 2.

На рисунке \ref{img:result} представлен пример работы программы, а на рисунке \ref{img:code} -- код программы.

\textit{Примечание:} поскольку отладочной платы на всех не хватало, прикладываю резултат работы программы одногруппника, который было разрешено использовать в собственном отчете.

\imgScale{0.3}{result}{Результат тестирования на отладочной плате}
\imgScale{0.3}{code}{Код программы}
