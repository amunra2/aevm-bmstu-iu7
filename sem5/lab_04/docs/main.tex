\chapter{Теоритические основы}
При выполнении лабораторной работы будет использоваться усоритель вычислений на \textbf{Xilinx Alveo}.

\section{Технология разработки ускорителей вычислений на Xilinx Alveo}

Ускорителями вычислений принято называть специальные аппаратные устройства, способные выполнять ограниченный ряд задач с большей параллельностью и за меньшее время в сравнении с универсальными микропроцессорными ЭВМ . Как правило, ускоритель представляет собой структуру, включающую большое количество примитивных микропроцессорных устройств, объединенных шинами связей.

Создание ускорителей вычислений является трудоемким процессом, так как охватывает не только аппаратную разработку самого устройства, но и предполагает оптимизацию архитектуры ЭВМ для обеспечения наибольшей пропускной способности каналов передачи операндов и результатов, а также минимизации задержек и вычислительных затрат при ожидании работы ускорителей. Можно условно разделить ускорители на два класса: ускорители на основе СБИС и на основе ПЛИС.

В данной лабораторной работе мы изучим технологию создания ускорителей вычислений на основе ПЛИС. Основной плат ускорителя \textbf{Xilinx Alveo U200} является ПЛИС xcu200-fsgd2104-2-e архитектуры Xilinx UltraScale, выполненная по 16-нанометровой технологии. Плата обеспечивает взаимодействие с хост-системой через интерфейс PCIe gen3 x16, и помимо ПЛИС содержит 4 планки памяти DIMM DDR4 по 16 ГБ, и два QSFP разъема для подключения 100ГБ Ethernet сети.

Для работы с ускорительной платой разработано специальное окружение \textbf{XRT} (Xilinx Runtime), включающее компоненты пользовательского пространства и драйвера ядра. XRT поддерживает как карты ускорителей на основе PCIe, так и встроенную архитектуру на основе MPSoC (для встраиваемых плат с ПЛИС Xilinx), обеспечивающую стандартизованный программный интерфейс для \textbf{Xilinx FPGA}.


\section{Описание архитектуры разрабатываемого ускорителя}

В ходе лабораторной работы будет использован базовый шаблон так называемого RTL проекта \textbf{VINC}, который может быть создан в \textbf{IDE Xilinx Vitis} и \textbf{САПР Xilinx Vivado}. Шаблон VINC выполняет попарное сложение чисел исходного массива и сохраняет результаты во втором массиве. Проект VINC включает:

\begin{itemize}
	\item проект ПО хоста, выполняющий инициализацию аппаратного ядра и его тестирование через OpenCL вызовы;
	\item синтезируемый RTL проект ядра ускорителя на языках Verilog и SystemVerilog;
	\item функциональный тест ускорителя VINC на языке SystemVerilog.
\end{itemize}

Проект VINC представляет собой аппаратное устройство, связанное шиной AXI4 MM (Memory mapped) с DDR[i] памятью, и получающее настроечные параметры по интерфейсу AXI4 Lite от программного обеспечения хоста (на рисунке \ref{img:project_on_plis}). В рамках всей системы используется единое 64-х разрядное адресное пространство, в котором формируются адреса на всех AXI4 шинах.

\imgScale{0.3}{project_on_plis}{Размещение проекта на ПЛИС xcu200-fsgd2104-2-e карты Alveo U200}


В каждой карте U200 имеется возможность подключить ускоритель к любому DDR[i] контроллеру в том регионе, где будет размещен проект. Всего для пользователя доступны 3 динамических региона: SLR0,1,2, для которых выделены каналы локальной памяти DDR[0], DDR[2], DDR[3] соответственно. Вся подключенная память DDR[0..3] доступна со стороны статического региона, в котором размещена аппаратная часть XRT.

Выбор одного из регионов для размещения проектов осуществляется на этапе так называемой линковки конфигурационного файла при помощи компилятора v++(фактически: компоновки, размещение и трассировки нескольких проектов в единый конфигурационный файл).


\chapter{Выполнение лабораторной работы}

Для изучения технологии будут выполнены следующие задания.

\section{Моделирование исходного проекта VINC}

На рисунке \ref{img:diag1_example} представлена транзакция чтения данных вектора на шине AXI4 MM из DDR памяти. Также на рисунке \ref{img:diag2_example} -- транзакция записи результата инкремента данных на шине AXI4 MM, а на рисунке \ref{img:diag3_example} инкремент данных в модуле.

\imgScale{0.3}{diag1_example}{Транзакция чтения данных вектора на шине AXI4 MM из DDR памяти}
\imgScale{0.4}{diag2_example}{Транзакция записи результата инкремента данных на шине AXI4 MM}
\imgScale{0.4}{diag3_example}{Инкремент данных в модуле}

\clearpage


\section{Моделирование проекта VINC, измененного по индивидульному варианту}

В соответствии с индивидуальным вариантом (Вариант 19) нужно было реализовать в коде следующую функцию:

\begin{equation}
	R[i] = max(A[i], 3000)
\end{equation}


На рисунке \ref{img:code_19var} представлена реализация функции на языке Verilog, которая была вставлена в код проект. При этом использовалась константа $CONST = 3000$.

\imgScale{0.5}{code_19var}{Функция индивидуального варианта}


При этом на рисунке \ref{img:diag1_var19} представлена транзакция чтения данных вектора на шине AXI4 MM из DDR памяти. Также на рисунке \ref{img:diag2_var19} -- транзакция записи результата инкремента данных на шине AXI4 MM, а на рисунке \ref{img:diag3_var19} инкремент данных в модуле.

\imgScale{0.3}{diag1_var19}{Транзакция чтения данных вектора на шине AXI4 MM из DDR памяти}
\imgScale{0.4}{diag2_var19}{Транзакция записи результата инкремента данных на шине AXI4 MM}
\imgScale{0.4}{diag3_var19}{Инкремент данных в модуле}

\clearpage


\section{Линковка проекта}

Для линковки проекта компилятором \textbf{v++} используется конфигурационный файл \textbf{config.cfg}, который содержит основную информацию для работы компилятора, такую, как:
\begin{itemize}
	\item количество и условные имена экземпляров ядер;
	\item тактовая частота работы ядра;
	\item для каждого ядра: выбор области SLR (SLR[0..2]), выбор DDR (DDR[0..3]) памяти, выбор высокопроизводительной памяти PLRAM(PLRAM[0,1,2]).
	\item параметры синтеза и оптимизации проекта.
\end{itemize}

На рисунке \ref{img:config} представлен конфигурационный файл для данного проекта, в котором \textbf{SLR1} и \textbf{DDR[2]}, что соответствует индивидуальному варианту.

\imgScale{0.6}{config}{Конфигурационный файл}

\textit{Примечание}: листинги файлов \textbf{v++*.log} и \textbf{*.xclbin.info} приведены в приложении.


\section{Тестирование}

После успешной линковки проекта получается файл \textbf{*.xclbin}. Также нужно получить \textit{exe} файл \textbf{host\_example.cpp}, который будет использован при тестировании.

Но прежде в \textbf{host\_example.cpp} необоходимо изменить условие проверки. На рисунке \ref{img:host_example_var19} представлена изменная проверка, которая соответствует функции для индивидуального варианта.

\imgScale{0.5}{host_example_var19}{Измененная проверка при тестировании}

В итоге, на рисунке \ref{img:test} приведены результаты тестированияы утилитой \textbf{xgdb}. Все тесты пройдены успешно.

\imgScale{0.3}{test}{Результаты тестирования}
